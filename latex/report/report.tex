\documentclass{article}

\usepackage[utf8]{inputenc}
\usepackage{geometry}
\usepackage{graphicx}


\usepackage{hyperref}

\usepackage{float}

\delimitershortfall-1sp
\newcommand\abs[1]{\left|#1\right|}


\title{Visualización de Datos MedioAmbientales de Castilla y León}
\author{Sergio García Prado}


\begin{document}

	\begin{titlepage}
		\centering
		{\scshape\LARGE Universidad de Valladolid \par}
		\vspace{1cm}
		{\scshape\Large Diseño y Evaluación de Sistemas Interactivos\par}
		\vspace{1.5cm}
		{\huge\bfseries Análisis y Visualización de Datos MedioAmbientales de Castilla y León\par}
		\vspace{2cm}
		{\Large\itshape Sergio García Prado\par}
	
		\vfill
		Seguimiento del trabajo en: \par
		\href{https://github.com/garciparedes/DatosMedioambientalesCYL}{github.com/garciparedes/DatosMedioambientalesCYL}
		\vfill


		% Bottom of the page
		{\large \today\par}
	\end{titlepage}



	\newpage
		\tableofcontents
	\newpage


	\section{Introducción}

		\subsection{Motivación:}

			\paragraph{}
			El análisis consiste en la obtención de aspectos poco obvios que se ocultan en los datos debido a la gran cantidad de ellos que se va a manejar. Los datos han sido obtenidos a partir de la iniciativa de Datos Abiertos de la Junta de Castilla y León, concretamente estos se refieren a Datos Medioambientales.
		
		
		\subsection{Objetivos:}
		
		
		\subsection{Estructura:}

	\section{Análisis de los datos}


		\subsection{Descripción}


		\subsection{Adquicisión}


		\subsection{Estructura de los datos}
			
			\paragraph{}
			Los datos que vamos a manejar se encuentran en fichero CSV, el cual a pesar de tener una estructura eficiente para ser manejado por máquinas, complica la lectura y extracción de información para los humanos. Además la cantidad de datos contenida en este es elevada (más de 5000 elementos).
			
			\paragraph{}
			A pesar de estas dificultades si que podemos ver la cabecera del fichero, la cual nos da información de la estructura del mismo. Además gracias a la herramienta D3.js se ha obtenido información acerca de ellas. El fichero tiene las siguientes entradas:
	
			\begin{itemize}
			
				\item {\bf Indicador } Contiene información sobre a lo que se refiere el dato. Representa una variable de tipo cualitativo nominal. Tras realizar una extracción de datos con D3 hemos obtenido que hay 49 indicadores distintos. Dado que son muchos lo que habrá que hacer es tratar de seleccionar los que más se adecuen a lo que queremos mostrar dado que mostrar datos demasiado heterogéneos y sin relación dificulta la visión de lo que se pretende mostrar.
				
				\item {\bf Provincia } Provincia a la que está referida el dato. Este campo es de tipo cualitativo nominal. Se refiere a la segmentación territorial de la comunidad autónoma. Como sabemos, Castilla y León está dividida en 9 provincias. Conviene destacar que hay indicadores que no poseen información para todas las provincias, como es el caso del "Tráfico aéreo de pasajeros", que suponemos restringe la información a las provincias que disponen de aeropuerto. Este hecho es algo a tener en cuenta cuando representemos los datos.

				\item {\bf Fecha Validez } Fecha hasta la cual el dato se toma como válido. Representa una variable cuantitativa. El motivo es que es una variable temporal. A pesar de ello toma únicamente valores discretos (años) Esto es algo a tener en cuenta cuando elijamos la manera en que lo representaremos.

				\item {\bf Valor } Valor que toma el dato respecto de su unidad.  Representa una variable cuantitativa de tipo continuo. Este campo es el que contiene la "información valiosa" del dato, ya que es la unidad que se utilizará como medida. Se representa en forma de número racional positivo.

				\item {\bf Unidad } Este campo nos proporciona la Unidad de Medida del dato, lo cual es algo muy importante a la hora de mostrar información.

				\item {\bf Frecuencia } Este campo representa la periodicidad con la que se ha obtenido el dato. La mayoría de los datos se han obtenido anualmente, aunque también existen algunos que se han obtenido bianualmente, como es el caso de los "Espacios Naturales".
			
			\end{itemize}
			
			
			
			
			
			\paragraph{}
			

		\subsection{Calidad}
		
		\subsection{Mejoras}
		
		\subsection{Transformaciones}


	\section{Planificación de la información}
	
		\subsection{Público Objetivo}
		

		\subsection{Propósito de la visualización:}
	
			\paragraph{}
			El proyecto de visualización se va a enfocar en el sector energético. Se pretende sacar conclusiones sobre cómo ha afrontado el territorio el cambio hacia nuevos métodos de producción energética con respecto al paso del tiempo.


		\subsection{Factores que afectan al proyecto:}


		\subsection{Enfoque:}


	\section{Diseño}
	
		\subsection{Datos a usar:}
	
	
		\subsection{Forma de representación:}
	
	
		\subsection{Decisiones de presentación:}


	\section{Implementación}

		\subsection{Bocetos:}

	
	\section{Conclusiones}
	
		\subsection{Conclusiones:}
	
	
		\subsection{Aspectos a Resaltas:}
	
	
		\subsection{Cuestiones no tenidas en cuenta:}



\end{document}
